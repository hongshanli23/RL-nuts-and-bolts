\documentclass{article}
\usepackage{macro}
\usepackage{graphicx}
\graphicspath{ {./} }
\title{Policy Gradient and REINFORCE}
\author{Hongshan Li(hongshal@amazon.com)}
\begin{document}
\maketitle
Last time we discussed policy gradient and how it is related to the relative
importance of the state and state action value

\begin{theorem}{Policy Gradient Theorem}
For a parametrized policy $\pi_{\theta}$, let $J(\theta)$ denote the
value of the policy $v_{\pi}(s_0)$, where $s_0$ is the initial 
state of the MDP. Then
\[
  \tag{1}
  \nabla J(\theta) \propto \int_{\mathbb{S}}\mu(s)
  \int_{\mathbb{A}}\nabla \pi(a|s)q_{\pi}(s, a) dsda
\]
where $\mathbb{S}$ denote the state space of the MDP and $\mathbb{A}$
denote the action space of the policy.
\end{theorem}
The theorem says the policy gradient, i.e. the direction that improves
the policy should be propotional to the importance of the state and 
the state action value. This should make intuitive sense. The gist
of policy gradient is that it maximizes the likelihood of good action,
and "good action" means action with high state action value at important
state. 

\section{Estimate Policy Gradient with Monte Carlo Method}
In order to estimate the policy gradient with Monte Carlo sampling,
the first thing we need to do is to write the integration in the policy
gradient theorem into an expectation over policy runs. So that we 
estimate it by executing the policy a couple of times in the env. 

Recall to compute the expectation of a function $f(x)$ over certain 
distribution $p(x)$ of $x \in \R$, we have 
\[
  \E_{x\sim p(x)}[f(x)] = \int_{\R} p(x)f(x) dx
\]
Therefore to make Eq 1 an expection over the policy, we do
\begin{align*}
  \nabla J(\theta) & \propto \int_{\mathbb{S}}\mu(s)
   \int_{\mathbb{A}}\nabla \pi(a|s)q_{\pi}(s, a) dsda \\
  & = \int_{\mathbb{S}}\mu(s)\int_{\mathbb{A}}\pi(a|s) \frac{\nabla\pi(a|s)}{\pi(a|s)}q_{\pi}(s, a) dsda \\
  & = \E_{\tau \sim \pi}(\frac{\nabla \pi(a|s)}{\pi(a|s)}q_{\pi}(s,a)) \\
  & = \E_{\tau \sim \pi}(\nabla \log(\pi(a|s))q_{\pi}(s,a)) 
\end{align*}

Recall 
\[
  \frac{d \log f(x)}{dx} = \frac{1}{f(x)} \frac{d f(x)}{dx}
\]

This trick of estimating gradient of through its log derivative is called 
\emph{log derivative trick} in statistical learning. 

This means by following the current policy $\pi$ for a few epsidoes and compute
the average of $\nabla \log(\pi(a|s))q_{\pi}(s,a)$ we would get an unbiased 
estimate of the policy gradient. 

\section{REINFORCE}
Now, we have enough ingredients for a vanilla policy gradient algorithm: REINFORCE
(REward Increment = nonnegative factor $\times$ Offset Reinforcement $\times$ Characteristic Eligibility)
\[
  \stackrel{\text{reward increment}}{\epsilon \nabla(J(\theta))} = \stackrel{\text{non-negative factor}}
  {\alpha}\times \stackrel{\text{offset reinforcement}}{q_{\pi}(s,a)} 
  \times \stackrel{\text{characteristic eligibility}}{\nabla \log \pi(a|s)}
\]

\begin{algorithm}[H]
  \caption{REINFORCE}
\end{algorithm}
\begin{algorithmic}[1]
  \STATE: INPUT: a differentiable policy $\pi_{\theta}$;
  a learning rate $\alpha$; 
  \STATE Initialze the parameters $\theta$
  
  \REPEAT
    \STATE Generate an episode $s_0, a_0, r_1, \cdots, s_{T-1}, a_{T-1}, r_T$
    \FOR {$t=T-1, T-2, \cdots, 0$}
    \STATE Compute 
    \[
      q_{\pi}(a_t, s_t)  = R_{t+1}+\gamma R_{t+2} + \cdots + 
        \gamma^{T-t-1}R_T
    \]
    \STATE Estiamte policy gradient
    \[
      \hat {\nabla J(\theta)} = \nabla \log \pi(a_t|s_t)q_{\pi}(a_t, s_t)
    \]
    \STATE Update parameter
    \[
      \theta \leftarrow \theta + \alpha\frac{1}{T}\hat{\nabla J(\theta)}
    \]
    \ENDFOR
  \UNTIL{Policy is good enough}
\end{algorithmic}

      

    


\end{document}


