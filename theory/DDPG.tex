\documentclass{article}
\usepackage{macro}
\begin{document}

Deep Deterministic Policy Gradient

Limitation of discretize action space for high dimentional control tasks:

1. Curse of dimensionality: with $n$ degree of freedom, if we discretize
each action space into $m$ discrete buckets, then the total number of 
buckets would be $m^n$.

2. Insufficient exploration. By limiting the number of discrete actions
we will underexplore the action space. 


Key ideas:
1. Stability tricks from DQN

2. Batch norm to normalize feature units (speed, angle have different 
units)
In supervised learning, BN is used to minimize covariance shift during 
training, by ensuring each layer receives whitend input. 
In DDPG, normalization is used on state input and all layers of the 
$\mu$ network (policy) and all layers of the $Q$ network prioir to 
the action input. 

Architecture of the $Q$ network:
Action can be concatenated in the middle. 

Exploration:

One advantage of off-policies algorithms such as DDPG is that we can treat 
the problem of exploration independently from learning problem. (what does 
it mean? )
We can construct an exploration policy $\mu^{\prime}$ 
\[
    \mu^{\prime}(s_t) = \mu(s_t | \theta^{\mu}_t) + \mathsrc{N}
\]

Add random process to action

\textbf{Ornstein-Uhlenbeck Process} This process depends on where the actor
is in the env. The deeper down into the env, the less random this process
becomes. 

Implementation details:
1. Rescale action into [-1, 1], 
2. Normalize return (can be false)
3. Normalize observation. Get the observation to [0, 1].
4. Rescale the reward. It is similar to my `reward_transform` function
5. L2 regularization of policy and value nets
6. Clip gradient
7. Noise is added when actor takes an action
8. Action is also clipped
9. Keep moving average of return, observation
10. Don't forget to scale the action back when use it for env
11. target network is updated in each 
12. don't use parallel env (for easy debugging) 
13. if the algorithm works in single env works, jazz up using MPI; 
each MPI process has its own env

Suppose we have a perfect $Q$-value estimator $Q_{\theta}(s, a)$, then
can we use it to learn a policy with off-policy alogorithm. This is 
because for an optimal policy $\eta : S \rightarrow A$, we have 
\[
    Q(s, \eta(s)) \geq Q(s, a) \forall a \in A
\]
Hence, we can update the policy parameter by doing gradient ascend on
\[
    \E[Q(s, \eta(s))]
\]
through $a$ variable. 


The idea of DDPG is to learn a $Q$-function via $Q$-learning styled 
algorithm and learn a deterministic policy $\eta$ by maximizing 
$Q(s, \eta(s))$. 

Why $\eta$ needs to be a deterministic policy? 

\begin{algorithm}[H]
\caption{Deep Deterministic Policy Gradient}
\label{alg}
\end{algorithm}

\begin{algorithmic}
\STATE Input: initial policy parameter $\theta$, $Q$-function 
parameters $\phi$, emtpy replay buffer $D$
\STATE Set target parameters equal to main parameters 
$\theta_{targ} \leftarrow \theta$, $\phi_{targ} \leftarrow \phi$

\REPEAT
\STATE Observe state $s$ and select an action (with Gaussian noise)
$a = clip(\eta_{\theta}(s) + \epsilon, a_{low}, a_{high}), 
\epsilon \sim N$
\STATE Execute $a$ in the environment
\STATE Observe next state $s^{\prime}$, reward $r$ and done signal 
$d$ signal $d$ to indicate whether $s^{\prime}$ is terminal.
\STATE Store $(s, a, r, s^{\prime}, d)$ in the replay buffer $D$
\STATE If $s^{\prime}$ is terminal, reset environment state 
\IF{it's time to update}
\FOR{however many updates}
\STATE Randomly sample a batch of transitions, 
$B = \{(s, a, r, s^{\prime}, d)\}$ from $D$
\STATE Compute targets
\[
    y(r, s^{\prime}, d) = r + \gamma(1-d)Q_{\phi_{targ}}(
        s^{\prime}, \mu_{\phi_{targ}}(s^{\prime})
        )
\]

\STATE Update policy by one sttep of gradient ascent using
\[
\nabla_{\phi}\frac{1}{|B|}\sum_{(s,a,r,s^{\prime},d)\in B}
(Q_{\phi}(s, a) - y(r, s^{\prime}, d))^2
\]
\STATE Update policy by one step of gradient using
\[
    \nabla_{\theta} \frac{1}{|B|}\sum_{s\in B} 
        Q_{\theta}(s, \mu_{\theta}(s))
\]
\STATE Update target network with 
\begin{align}
    \phi_{targ} \leftarrow \rho \phi_{targ} + (1-\rho)\phi \\
    \theta_{targ} \leftarrow \rho \theta_{targ} + (1-\rho)\theta
\end{align}
\ENDFOR
\ENDIF
\UNTIL{convergence}

\end{algorithmic}


\end{document}
