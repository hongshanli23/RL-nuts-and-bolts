\documentclass{article}
\usepackage[T1]{fontenc}
\usepackage[utf8]{inputenc}

\usepackage{macro}

\begin{document}

SAC is a off-policy alogrithm. 
SAC trains the agent by adding entropy of the action into reward. The 
optimization problem becomes 
\[
    \pi^{*} = \argmax_{\pi} \E_{\tau \sim \pi}[
        \sum_{t=0}^{\infty}\gamma^t(R(s_t, a_t, s_{t+1}) + \alpha 
        H(\pi(\cdot | s_t)))]
\]

The value function of the states becomes
\[
    V^{\pi}(s) = \E_{\tau\sim\pi}[
        \sum_{t=0}^{\infty} \gamma^t (
        R(s_t, a_t, s_{t+1}) + \alpha H(\pi(s))
        )]
\]

The state action value is changed to include the entropy at each time
step except the first one.  

\[
    Q^{\pi}(s, a) = \E_{\tau\sim\pi} [
        \sum_{t=0}^{\infty}\gamma^t(
        R(s_t, a_t, s_{t+1}))
        + \sum_{t=1}^{\infty}\alpha \gamma^t H(\pi(s_t)) 
        | s_0=s, a_0=a]
\]

With these definition of $V$ and $Q$, we have
\[
    V^{\pi}(s) = \E_{\sprime\in P}[R(s, a, \sprime) + \gamma V^{\pi}(\sprime)]
\]

SAC learns a policy $\pi_{\theta}$ and two $Q$-functions $Q_{\phi_1}$ and 
$Q_{\phi_2}$. 

Similarities to DDPG and TD3
\begin{itemize}
    \item both $Q$-functions are learned with MSE minimization, by regressing to a single shared target.
    \item the shared target is computed using target $Q$-networks and 
        polyak interpolation,
    \item \textbf{clipped double-$Q$} trick is used to avoid 
        overconfidence
\end{itemize}

Differences to DDPG and TD3
\begin{itemize}
    \item target includes term from SAC's entropy regularization
    \item next state actions used in target come from \textbf{current}
        policy
    \item no explicit policy smooth (in DDPG and TD3 random noise is 
        added to action for exploration). SAC trains a stochastic
        policy.
\end{itemize}

The entropy term and next state action term in $Q^{\pi}$ are computed
using the \textbf{current policy}. The next state $\sprime$ is sampled
from the replay buffer. But this means additional computed
is need to sample the next action. 

The estimate of $Q$-value (based one sample from experience replay) is
\[
    Q^{\pi} = r + \gamma(Q^{\pi}(\sprime, \tilde \aprime) 
    - \alpha \log \pi(\tilde \aprime | \sprime)), 
    \tilde \aprime \sim \pi(\cdot | \sprime)
\]

We use $\tilde \aprime$ to emphasize that next state action has to come
from the current policy. 

\textbf{Update the Q-function}

SAC trains 2 $Q$-function targets, the smaller one is used to compute
the sample backup. (like in TD3)

The loss for the $Q$-networks in SAC are:
\[
    L(\phi_i, \D) = \E_{(s,a,r,\sprime,d)\sim\D)}[
        (Q_{\phi_i}(s,a)-y(r,\sprime,d))^2]
\]
where
\[
    y(r,\sprime,d)=r + \gamma(1-d)(\min_{j=1,2}
    Q_{\phi_{targ,j}}(\sprime, \tilde \aprime) - 
    \alpha \log \pi_{\theta}(\tilde \aprime | \sprime)
    )
\]

\textbf{Update the policy}

The policy should, in each state, act to maximize the expected future
return plus the expected entropy,i.e. it should maximize $V^{\pi}(s)$
\begin{align*}
    V^{\pi}(s) = \E_{a \sim \pi}[Q^{\pi}(s,a)] + \alpha H(\pi(\cdot|s)) \\
    & = \E_{a \sim\pi}[Q^{\pi}(s,a) - \alpha\log\pi(a|s)]
\end{align*}

Use \textbf{reparametrization trick} for policy optimization, in which
a sample from $\pi(\cdot | s)$  is drawn by computing a deterministic
function of state, policy params, and independent noise. We squash 
the Gaussian policy through $\tanh$ fun
\[
    \tilde a_{\theta}(s,\zeta) = \tanh(\mu_{\theta}(s) + 
    \sigma_{\theta}(s)\odot \zeta), \zeta \sim N(0, 1)
\]

1. $\tanh$ changes the distribution of policy (no longer Gaussain),
but one can still compute the log probability. If we don't squash
the Gaussian dist, then it could have infinite entropy. 

2. The standard deviation depends on state. In TRPO, PPO standard 
deviation can be independent from the state (verity it. In my 
implementation, std depends on input state). 
According to openai, SAC does not work if the std is independent from
the input state. 
If entropy is part of learning objective, then naturally std should be
a function of input state, because entropy is a function of std. 

\todo{experiment with constant std}

Why reparametrization is useful?
It let us write expectation over $\zeta$ rather than action, which 
depends on the policy parameters.
\[
    \E_{a\sim\pi_{\theta}}[Q^{\pi_{\theta}}(s,a) - 
    \alpha\log\pi_{\theta}(a|s)] 
    = \E_{\zeta\sim N}[Q^{\pi_{\theta}}(s, \tilde a(s, \zeta))
    - \alpha\log \pi_{\theta}(\tilde a(s,\zeta)|s)]
\]

final step to get the policy loss is to replace $Q^{\pi_{\theta}}$
with one of our function approximators $Q_{\phi_i}$ (the less confident
one). The optimization objective becomes
\[
    \max_{\theta}\E_{s\sim\D,\zeta\sim N}[
        \min_{j=1,2}Q_{\phi_j}(s, \tilde a_{\theta}(s,\zeta))
        - \alpha\log \pi_{\theta}(\tilde a_{\theta}(s,\zeta)|s)
    ]
\]
it is almost similiar to DDPG's training objective. 



Questions:

Why in target loss, the next state action needs to be sampled from the 
current policy? The intuition is with better policy, can you make a 
better estimate of state action value. But it behaves like a on-policy
algorithm.


\begin{algorithm}[H]
\caption{Soft Actor-Critic}
\label{alg}
\end{algorithm}

\begin{algorithmic}
% \STATE Input: initial policy parameter $\theta$, $Q$-function 
% parameters $\phi$, emtpy replay buffer $D$
% \STATE Set target parameters equal to main parameters 
% $\theta_{targ} \leftarrow \theta$, $\phi_{targ} \leftarrow \phi$
% \REPEAT
% \FOR
% \ENDFOR
% \IF
% \ENDIF
% \UNTIL{convergence}
\STATE Input: initial policy params $\theta$, $Q$-function params 
$\phi_1$ and $\phi_2$, empty replay buffer $\D$.
\STATE Set target parameters equal to main parameters 
    $\phi_{targ,1}\leftarrow \phi_1, \phi_{targ,2}\leftarrow \phi_2$
\REPEAT
    \STATE observe state $s$ and select action 
    $a\sim\pi_{\theta}(\cdot|s)$
    \STATE execute $a$ in the env
    \STATE observe next state $\sprime, r, d$ signal from env
    \STATE store $(s,a,r,\sprime,d)$ in the buffer
    \STATE if $\sprime$ is terminal, then reset env
    \IF {it's time to update}
        \FOR{j in range(num of updates)}
            \STATE randomly sample a batch of transitions, 
            $B=\{(s,a,r,\sprime,d)\}$ from $\D$
            \STATE compute target for $Q$ functions
            \[
                y(r,\sprime,d) = r + \gamma(1-d)(\min_{i=1,2}
                Q_{\phi_{targ,i}}(\sprime,\tilde\aprime)
                - \alpha \log \pi_{\theta}(\tilde\aprime|\sprime)),
                \tilde \aprime \sim \pi_{\theta}(\cdot|\sprime)
            \]
            \STATE update $Q$-fn by one step of gradient descent
            \[
                \nabla_{\phi_i}\frac{1}{|B|}
                \sum_{(s,a,r,\sprime,d)\in B}
                (Q_{\phi_i}(s,a) - y(r,\sprime,d))^2
            \]
            \STATE Update policy by one step gradien ascent
            \[
                \nabla_{\theta}\frac{1}{|B|}
                \sum_{s\in B}(\min_{i=1,2}
                Q_{\phi_i}(s,\tilde a_{\theta}(s)) - 
                \alpha\log\pi_{\theta}(\tilde a_{\theta}(s)|s))
            \]
            where $\tilde a_{\theta}(s)$ is a sample from
            $\pi_{\theta}(\cdot|s)$ which is differentiable wrt $\theta$
            via reparam trick.
            \STATE update target networks
            \[
                \phi_{targ,i}\leftarrow \rho\phi_{targ,i} + 
                (1-\rho)\phi_i
            \]
        \ENDFOR
    \ENDIF
\UNTIL{convergence}
\end{algorithmic}

\textbf{Reparametrization Trick}

\textbf{Squashing function} The policy network outputs mean and std of a
Gaussian distribution, where the action is sampled from. 
\[
    a = m + \sigma * \zeta, \zeta \sim N(0, 1)
\]
One trick to make the training more stable is to squash it into [0,1] 
via $\tanh$ function. 

Let $u \in R^{D}$ be a random variable with density $\mu(u|s)$, what is 
the density of $a = \tanh(u)$?

The more general framework is how the same function changes under 
change of coordinate. 
Let $x$ be a coordinate system on $\R^n$ and let $f$ be a function on 
$\R^n$ represented through $x$. Now, consider a change of coordinate
$y = y(x)$. Then what is the new representation of $f$ in $y$?

Review differential calculus on manifold.

\end{document}
    
